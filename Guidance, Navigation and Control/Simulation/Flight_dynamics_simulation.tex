\documentclass{article}
\usepackage{graphicx} % Required for inserting images
\usepackage[a4paper, total={6in, 10in}]{geometry}
\usepackage{gensymb}

\title{Project Viikate Flight Dynamics Simulation}
\author{Taneli Juutinen}


\begin{document}

\maketitle

\section{Introduction}
This document explains the work done to simulate the flight dynamics of a small aerodynamically controlled rocket. First, the mathematical model for simulation is laid out in section 2. Section 3 explains the implementation of this mathematical model in the matlab/simulink environment.

\section{Mathematical model}
\subsection{Canards}
Canards are fins positioned in the front section of a rocket. In this case, the canard angle relative to the rocket body is actively controlled so we are able to influence the flight path of the rocket. When canards are rotated relative to local air speed, they produce a lift force in the direction perpendicular to the roll axis of the rocket. The lift force of a fin can be calculated as a function of local air speed and the coefficient of lift of the fin.
\begin{equation}
    F_L=\frac{1}{2}\rho C_LU^2
\end{equation}
where $\rho$ is the air density, $C_L$ is the coefficient of lift, and U is the free-stream velocity in the direction of the fin chord line at $0\degree $ angle of attack (AoA). The lift force produces a torque on the rocket proportional to its distance from the rotation axes. 
\begin{equation}
    \mathbf{M}_c=\mathbf{r}_c\times \mathbf{F}_L
\end{equation}
where $\mathbf{r}_c$ is the position vector from center of mass to the force location. 

An important variable of canard lift is the angle of attack relative to the air stream. To obtain the local AoA of a canard fin, we need to know the angle relative to the rocket body, and the AoA of the rocket body.
\begin{equation}
    \alpha = \arctan2(\mathbf{n}\mathbf{V}_b,\mathbf{c}\mathbf{V}_b)
\end{equation}
where $\mathbf{n}$ is the normal vector of a canard fin pointing perpendicular to the chord line, $\mathbf{V}_b$ is the rocket velocity in a body-fixed coordinate frame, and $\mathbf{c}$ is the chord line pointing vector. 

\section{Matlab implementation}

\end{document}